\ProvidesPackage{preamble}

% From IEEEtran template (might be cleaned up)
% \usepackage{cite}
\usepackage{amsmath,amssymb,amsfonts}
\usepackage{algorithmic}
\usepackage{graphicx}
\usepackage{textcomp}
\usepackage{xcolor}

%% ------------------------------------------------------------------------------%
%% Miscelaneous packages
%% ------------------------------------------------------------------------------%

\usepackage{enumitem}
\usepackage{xspace}
\usepackage{subcaption}
\captionsetup[subfigure]{format=hang}
\usepackage[T1]{fontenc}

%% ------------------------------------------------------------------------------%
%% Colors
%% ------------------------------------------------------------------------------%

\definecolor{backcolour}{rgb}{0.95,0.95,0.92}
\definecolor{bluekeywords}{rgb}{0.13,0.13,1}
\definecolor{codegray}{gray}{0.9}
\definecolor{ForestGreen}{RGB}{0,200,0}
\definecolor{mygreen}{rgb}{0,0.6,0}
\definecolor{mygray}{rgb}{0.5,0.5,0.5}
\definecolor{mymauve}{rgb}{0.58,0,0.82}
\definecolor{myteal}{RGB}{0,115,115}
\definecolor{redstrings}{rgb}{0.9,0,0}

%% ------------------------------------------------------------------------------%
%% Listings and colors
%% ------------------------------------------------------------------------------%

\usepackage{listings}
\lstset{
  backgroundcolor=\color{backcolour},
  basicstyle=\ttfamily\footnotesize,
  breakatwhitespace=false,
  breaklines=true,
  captionpos=b,
  commentstyle=\color{mygreen},
  deletekeywords={...},
  % escapeinside={\%*}{*)},
  escapeinside={(*@}{@*)},
  extendedchars=true,
  firstnumber=0,
  % frame=single,
  keepspaces=true,
  keywordstyle=\color{bluekeywords}\bfseries,
  language=C++,
  morekeywords={*,include,...},
  numbers=left,
  numbersep=5pt,
  numberstyle=\tiny\color{mygray},
  rulecolor=\color{black},
  showspaces=false,
  showstringspaces=false,
  showtabs=false,
  stepnumber=1,
  stringstyle=\color{redstrings},
  tabsize=2,
  title=\lstname
}

%% ------------------------------------------------------------------------------%
%% Glossary
%% ------------------------------------------------------------------------------%


%% \usepackage[style=long3colheaderborder,acronym,automake,footnote,toc]{glossaries}
%% https://www.dickimaw-books.com/gallery/index.php?label=mixed-glossary-emph2
%% style=tree, stylemods=mcols

% Glossaries-extra
% \usepackage[automake,stylemods=mcols,toc]{glossaries-extra}
% \usepackage{glossaries-extra}
% \newglossary[slg]{symbol}{sym}{sdn}{List of Symbols}
% \setabbreviationstyle[acronym]{long-postshort-user}
% \setabbreviationstyle[slg]{long-postshort-user}
% \renewcommand{\glsxtrregularfont}[1]{\textcolor{violet}{\glsxtrifwasfirstuse{{#1}}{#1}}}
% \renewcommand{\glsxtrabbreviationfont}[1]{{\glsxtrifwasfirstuse{{#1}}{#1}}}
% \renewcommand{\glsfirstabbrvuserfont}[1]{{#1}}

% \renewcommand*{\glsxtrsetupfulldefs}{%
%   \renewcommand*{\glsxtrifwasfirstuse}[2]{##2}%
% }

%% Other styles are:
%% \setabbreviationstyle[acronym]{short-sc-postfootnote}
%% \setabbreviationstyle[acronym]{long-em-short-em}
%% \setabbreviationstyle[acronym]{short-footnote}
%% \setabbreviationstyle[acronym]{long-short-user}

%% \renewcommand*{\glsfirstlonguserfont}[1]{\emph{#1}}
%% \renewcommand{\glsxtrregularfont}[1]{\textcolor{violet}{\glsxtrifwasfirstuse{{#1}}{#1}}}
%% \renewcommand{\glsxtrabbreviationfont}[1]{\textcolor{myteal}{\glsxtrifwasfirstuse{{#1}}{\texttt{#1}}}}
%% \renewcommand{\glsxtrabbreviationfont}[1]{\textcolor{myteal}{\glsxtrifwasfirstuse{}{\let\glsfirstlonguserfont\glslonguserfont}#1}}
%% \renewcommand{\glsxtrabbreviationfont}[1]{\textcolor{myteal}{\glsxtrifwasfirstuse{{{\let\glsfirstlonguserfont\glslonguserfont}#1}}{\texttt{#1}}}}

\usepackage{glossaries}
% \makeglossaries

%% ------------------------------------------------------------------------------%
%% Command definitions
%% ------------------------------------------------------------------------------%

\definecolor{codegray}{gray}{0.9}

\newcommand{\code}[1]{{\texttt{#1}}}

%% This package should be here! don't move it!
\usepackage{cleveref}

%% ------------------------------------------------------------------------------%
%% Macros
%% ------------------------------------------------------------------------------%

% MPI Functions
\def\id{\emph{id}\xspace}
\def\pmpi{PMPI\xspace}
\def\rank{\emph{rank}\xspace}
\def\ranks{\emph{ranks}\xspace}
\def\root{\emph{root}\xspace}
\def\tag{\emph{tag}\xspace}
\def\mpiallreduce{\code{MPI\_Allreduce}\xspace}
\def\mpialltoall{\code{MPI\_Alltoall}\xspace}
\def\mpiallgather{\code{MPI\_Allgather}\xspace}
\def\mpiallgatherv{\code{MPI\_Allgatherv}\xspace}
\def\mpianysource{\code{MPI\_ANY\_SOURCE}\xspace}
\def\mpianytag{\code{MPI\_ANY\_TAG}\xspace}
\def\mpiband{\code{MPI\_BAND}\xspace}
\def\mpibarrier{\code{MPI\_Barrier}\xspace}
\def\mpibcast{\code{MPI\_Bcast}\xspace}
\def\mpicomm{\emph{communicator}\xspace}
\def\mpifinalize{\code{MPI\_Finalize}\xspace}
\def\mpigather{\code{MPI\_Gather}\xspace}
\def\mpigatherv{\code{MPI\_Gatherv}\xspace}
\def\mpiiallgather{\code{MPI\_Iallgather}\xspace}
\def\mpiiallreduce{\code{MPI\_Iallreduce}\xspace}
\def\mpiinitthread{\code{MPI\_Init\_thread}\xspace}
\def\mpiirecv{\code{MPI\_Irecv}\xspace}
\def\mpiisend{\code{MPI\_Isend}\xspace}
\def\mpilonglongint{\code{MPI\_LONG\_LONG\_INT}\xspace}
\def\mpimpit{\code{MPI\_T}\xspace}
\def\mpirecv{\code{MPI\_Recv}\xspace}
\def\mpireduce{\code{MPI\_Reduce}\xspace}
\def\mpiscan{\code{MPI\_Scan}\xspace}
\def\mpiscatter{\code{MPI\_Scatter}\xspace}
\def\mpisend{\code{MPI\_Send}\xspace}
\def\mpithreadsingle{\code{MPI\_THREAD\_SINGLE}\xspace}
\def\mpithreadfunneled{\code{MPI\_THREAD\_FUNNELED}\xspace}
\def\mpithreadmultiple{\code{MPI\_THREAD\_MULTIPLE}\xspace}
\def\mpithreadserialized{\code{MPI\_THREAD\_SERIALIZED}\xspace}
\def\mpithreadult{\code{MPI\_THREAD\_ULT}\xspace}
\def\mpiwait{\code{MPI\_Wait}\xspace}
\def\mpiwaitall{\code{MPI\_Waitall}\xspace}
\def\mpiwaitany{\code{MPI\_Waitany}\xspace}
\def\mpiwin{\code{MPI\_Win}\xspace}

% Programming models, libraries, frameworks, technologies, algorithms, etc.
\def\alg`binary{\emph{Binary}\xspace}
\def\algbinomial{\emph{Binomial}\xspace}
\def\algchain{\emph{Chain}\xspace}
\def\algknomial{\emph{Knomial}\xspace}
\def\alglinear{\emph{Linear}\xspace}
\def\algring{\emph{Ring}\xspace}
\def\algrsa{\gls*{RSA}\xspace}
\def\algpairwise{\emph{Pairwise Exchange}\xspace}
\def\algpipeline{\emph{Pipeline}\xspace}
\def\algrab{\emph{Rabenseifner}\xspace}
\def\algrd{\emph{Recursive Doubling}\xspace}
\def\algrh{\emph{Recursive Halving}\xspace}
\def\algsegring{\emph{Segmented Ring}\xspace}
\def\algsynclinear{\emph{Synchornized Linear}\xspace}
\def\algtwoprocs{\emph{Two Processes}\xspace}
\def\allreduce{\emph{Allreduce}\xspace}
\def\api{\gls*{API}\xspace}
\def\apis{\gls*{API}s\xspace}
\def\beluga{B\'eluga\xspace}
\def\btl{\gls*{BTL}\xspace}
\def\coll{\emph{collective}\xspace}
\def\Coll{\emph{Collective}\xspace}
\def\colls{\emph{collectives}\xspace}
\def\Colls{\emph{Collectives}\xspace}
\def\cntk{CNTK\xspace}
\def\cpu{CPU\xspace}
\def\cpus{CPUs\xspace}
\def\cuda{\gls*{CUDA}\xspace}
\def\cudaipc{\gls*{CUDA} \gls*{IPC}\xspace}
\def\dag{\gls*{DAG}\xspace}
\def\d2h{\gls*{D2H}\xspace}
\def\dl{\gls*{DL}\xspace}
\def\dst{\emph{dst}\xspace}
\def\git{git\xspace}
\def\github{GitHub\xspace}
\def\google{Google\xspace}
\def\grep{\code{grep}\xspace}
\def\h2d{\gls*{H2D}\xspace}
\def\hct{\code{HOROVOD\_CYCLE\_TIME}\xspace}
\def\hpc{\gls*{HPC}\xspace}
\def\hft{\code{HOROVOD\_FUSION\_THRESHOLD}\xspace}
\def\horovod{Horovod\xspace}
\def\ib{\gls*{IB}\xspace}
\def\ipc{\gls*{IPC}\xspace}
\def\ipcs{\gls*{IPC}s\xspace}
\def\hashmap{\emph{HashMap}\xspace}
\def\keras{Keras\xspace}
\def\key{\emph{key}\xspace}
\def\linux{GNU/Linux\xspace}
\def\matex{MaTEx\xspace}
\def\msg{\emph{message matching}\xspace}
\def\Msg{\emph{Message matching}\xspace}
\def\ml{Machine Learning\xspace}
\def\mpi{\gls*{MPI}\xspace}
\def\mpich{MPICH\xspace}
\def\mpichucx{MPICH+\gls*{UCX}\xspace}
\def\mpicuda{\gls*{MPI}+\gls*{CUDA}\xspace}
\def\mpimpi{\gls*{MPI}+\gls*{MPI}\xspace}
\def\mpinccl{\gls*{MPI}+\gls*{NCCL}\xspace}
\def\mpiopencl{\gls*{MPI}+\gls*{OpenCL}\xspace}
\def\mpithree{\gls*{MPI}-3\xspace}
\def\mpix{\gls*{MPI}+X\xspace}
\def\mvapich{MVAPICH\xspace}
\def\mvapichtwo{MVAPICH2\xspace}
\def\mvapichtwogdr{MVAPICH2\gls*{GDR}\xspace}
\def\mxnet{MXNet\xspace}
\def\nic{\gls*{NIC}\xspace}
\def\nics{\gls*{NIC}s\xspace}
\def\nccl{\gls*{NCCL}\xspace}
\def\ncclrecv{\code{NCCL\_Recv}\xspace}
\def\ncclsend{\code{NCCL\_Send}\xspace}
\def\nccltwo{\gls*{NCCL}2\xspace}
\def\numa{\gls*{NUMA}\xspace}
\def\nvshmem{NVSHMEM\xspace}
\def\omb{\gls*{OMB}\xspace}
\def\omp{OpenMP\xspace}
\def\ompi{\gls*{OMPI}\xspace}
\def\opal{\gls*{OPAL}\xspace}
\def\opencl{\gls*{OpenCL}\xspace}
\def\openacc{\gls*{OpenACC}\xspace}
\def\openmpi{Open \gls*{MPI}\xspace}
\def\openmpiucx{Open \gls*{MPI}+\gls*{UCX}\xspace}
\def\openshmem{\gls*{OpenSHMEM}\xspace}
\def\os{\gls*{OS}\xspace}
\def\osc{\emph{one-sided}\xspace}
\def\Osc{\emph{One-sided}\xspace}
\def\pap{\gls*{PAP}\xspace}
\def\paps{\gls*{PAP}s\xspace}
\def\part{\emph{partitioned}\xspace}
\def\pat{\gls*{PAT}\xspace}
\def\pats{\gls*{PAT}s\xspace}
\def\prrte{\gls*{PRRTE}\xspace}
\def\pt{PyTorch\xspace}
\def\pthreads{\gls*{Pthreads}\xspace}
\def\ptp{\gls*{P2P}\xspace}
\def\ptps{\gls*{P2P}s\xspace}
\def\p2p{Peer-to-Peer\xspace}
\def\rl{Reinforcement Learning\xspace}
\def\sharp{SHARP\texttrademark\xspace}
\def\systemv{System V\xspace}
\def\src{\emph{src}\xspace}
\def\tf{TensorFlow\xspace}
\def\tfbench{\code{tf\_cnn\_benchmark}\xspace}
\def\temp{\emph{temp}\xspace}
\def\ucc{\gls*{UCC}\xspace}
\def\ucf{\gls*{UCF}\xspace}
\def\ucm{\gls*{UCM}\xspace}
\def\ucp{\gls*{UCP}\xspace}
\def\ucs{\gls*{UCS}\xspace}
\def\uct{\gls*{UCT}\xspace}
\def\ucx{\gls*{UCX}\xspace}

% GPUs, CUDA, & NVIDIA
\def\asynccopy{\gls*{AC}\xspace}
\def\amd{AMD\xspace}
\def\ampere{Ampere\xspace}
\def\cevent{\gls*{CUDA} \emph{event}\xspace}
\def\cevents{\gls*{CUDA} \emph{events}\xspace}
\def\cgroups{Cooperative Groups\xspace}
\def\cgraph{CUDA Graph\xspace}
\def\cgraphs{CUDA Graphs\xspace}
\def\ctx{\emph{context}\xspace}
\def\ctxs{\emph{contexts}\xspace}
\def\cstream{\gls*{CUDA} \emph{stream}\xspace}
\def\cstreams{\gls*{CUDA} \emph{streams}\xspace}
\def\cudadevsync{\code{cudaDeviceSynchronize}\xspace}
\def\cudadevreset{\code{cudaDeviceReset}\xspace}
\def\cudahostalloc{\code{cudaHostAlloc}\xspace}
\def\cudahostfunc{\code{cudaHostFunction}\xspace}
\def\cudamemregister{\code{cudaMemRegister}\xspace}
\def\cudamemalloc{\code{cudaMalloc}\xspace}
\def\cudamemadvise{\code{cudaMemAdvise}\xspace}
\def\cudamemcpy{\code{cudaMemcpy}\xspace}
\def\cudamemcpyasync{\code{cudaMemcpyAsync}\xspace}
\def\cudamemset{\code{cudaMemset}\xspace}
\def\cudamemprefetch{\code{cudaMemPrefetch}\xspace}
\def\cudastreamsync{\code{cudaStreamSynchronize}\xspace}
\def\dgxone{DGX-1\xspace}
\def\dgxtwo{DGX-2\xspace}
\def\event{\emph{event}\xspace}
\def\events{\emph{events}\xspace}
%\def\gpudirect{GPUDirect\textsuperscript{\textregistered}\xspace}
\def\gdr{\gls*{GDR}\xspace}
\def\gpu{\gls*{GPU}\xspace}
\def\gpus{\gls*{GPU}s\xspace}
\def\gpudirect{GPUDirect\xspace}
\def\gpudirectptp{GPUDirect\_\gls*{P2P}\xspace}
\def\gpudirectasync{GPUDirect\_Async\xspace}
\def\gpudirectrdma{GPUDirect\_\gls*{RDMA}\xspace}
\def\graphlaunch{\code{cudaGraphLaunch}\xspace}
\def\gstream{\gls*{GPU} \emph{stream}\xspace}
\def\gstreams{\gls*{GPU} \emph{streams}\xspace}
\def\host{\emph{host}\xspace}
\def\hyperq{Hyper-Q\xspace}
\def\kernel{kernel\xspace}
\def\kernels{kernels\xspace}
\def\mellanox{Mellanox\xspace}
\def\mig{\gls*{MIG}\xspace}
\def\mps{\gls*{MPS}\xspace}
\def\nvidia{NVIDIA\xspace}
\def\nvgroup{NVGroup\xspace}
\def\nvlink{NVLink\xspace}
\def\nvlinkctc{NVLink Chip-to-Chip (C2C)\xspace}
\def\nvlinkone{NVLink-V1\xspace}
\def\nvlinks{NVLinks\xspace}
\def\nvlinktwo{NVLink-V2\xspace}
\def\nvlinkthree{NVLink-V3\xspace}
\def\nvlinksli{\gls*{NVLink-SLI}\xspace}
\def\nvswitch{NVSwitch\xspace}
\def\pcie{\gls*{PCIe}\xspace}
\def\sm{\gls*{SM}\xspace}
\def\sms{\gls*{SM}s\xspace}
\def\stream{\emph{stream}\xspace}
\def\streams{\emph{streams}\xspace}
\def\volta{Volta\xspace}

% ACRONYMS
% \newacronym{NameInSort}{NameToShow}{Definition}
\newacronym{AC}{AC}{Asynchronous Copy}
\newacronym{ANL}{ANL}{Argonne National Laboratory}
\newacronym{ARMCI}{ARMCI}{Aggregate Remote Memory Copy Interface}
\newacronym{API}{API}{Application Programming Interface}
\newacronym{ATS}{ATS}{Address Translation Services}
\newacronym{BLAS}{BLAS}{Basic Linear Algebra Subprograms}
\newacronym{BTB}{BTB}{Binomial Tree Based}
\newacronym{BTL}{BTL}{Byte Transfer Layer}
\newacronym{BPMF}{BPMF}{Bayesian Probabilistic Matrix Factorize}
\newacronym{CHAMPION}{CHAMPION}{Communication-aware Hardware-Assisted MPI Overlap eNgine}
\newacronym{CISC}{CISC}{Complex Instruction Set Computing}
\newacronym{CNN}{CNN}{Convolutional Neural Networks}
\newacronym{CQ}{CQ}{Completion Queue}
\newacronym{CRI}{CRI}{Communication Resources Instance}
\newacronym{CTS}{CTS}{Clear To Send}
\newacronym{cuBLAS}{cuBLAS}{NVIDIA CUDA Blas Library}
\newacronym{CUDA}{CUDA}{Compute Unified Device Architecture}
\newacronym{CUDA-Capable}{CUDA-Capable}{Capable for programming in CUDA Platform}
\newacronym{cuFFT}{cuFFT}{NVIDIA CUDA Fast Fourier Transform Library}
\newacronym{cuDNN}{cuDNN}{NVIDIA CUDA Deep Neural Networks Library}
\newacronym{cuRAND}{cuRAND}{NVIDIA CUDA Random Number Generator Library}
\newacronym{cuSPARSE}{cuSPARSE}{NVIDIA CUDA Sparse Matrix Library}
\newacronym{D2D}{D2D}{Device-to-Device}
\newacronym{DAG}{DAG}{Directed Acyclic Graph}
\newacronym{D2H}{D2H}{Device-to-Host}
\newacronym{DL}{DL}{Deep Learning}
\newacronym{DMA}{DMA}{Direct Memory Access}
\newacronym{DNN}{DNN}{Deep Neural Networks}
\newacronym{DriverAPI}{Driver API}{Driver Application Programming Interface}
\newacronym{DSL}{DSL}{Domain-Specific Language}
\newacronym{EP}{EP}{Endpoints}
\newacronym{ETL}{ETL}{Extract, Transform and Load}
\newacronym{FDS}{FDS}{Fire Dynamics Simulator}
\newacronym{FLOPS}{FLOPS}{Floating-point Operations Per Second}
\newacronym{FFT}{FFT}{Fast Fourier Transform}
\newacronym{GCC}{GCC}{GNU Compiler Collection}
\newacronym{GDR}{GDR}{GPUDirect RDMA}
\newacronym{GPU}{GPU}{Graphics Processing Unit}
\newacronym{GPGPU}{GPGPU}{General Purpose Computing on Graphics Processing Unit}
\newacronym{GSB}{GSB}{GPU Shared Buffer}
\newacronym{H2D}{H2D}{Host-to-Device}
\newacronym{H2H}{H2H}{Host-to-Host}
\newacronym{HCA}{HCA}{Host Channel Adapter}
\newacronym{HDFS}{HDFS}{Hadoop File System}
\newacronym{HMCS}{HMCS}{Hierarchical MCS}
\newacronym{HMPI}{HMPI}{Hybrid MPI}
\newacronym{HOL}{HOL}{Head-Of-Line}
\newacronym{HOOMD-blue}{HOOMD-blue}{Highly Optimized Object-oriented Many-particle
  Dynamics - Blue Edition}
\newacronym{HPC}{HPC}{High-Performance Computing}
\newacronym{HPL}{HPL}{High-Performance Linpack}
\newacronym{IB}{IB}{InfiniBand}
\newacronym{IPC}{IPC}{Inter-Process Communication}
\newacronym{ILP}{ILP}{Instruction Level Parallelism}
\newacronym{ISA}{ISA}{Instruction Set Architecture}
\newacronym{IPO}{IPO}{Input Processor Output}
\newacronym{JCUDA}{JCUDA}{Java Bindings for CUDA}
\newacronym{JNI}{JNI}{Java Native Interface}
\newacronym{JVM}{JVM}{Java Virtual Machine}
\newacronym{LAMMPS}{LAMMPS}{Large-scale Atomic/Molecular Massively Parallel Simulator}
\newacronym{LBM}{LBM}{Lattice Boltzmann Method}
\newacronym{LLN}{LLN}{Low-Level Network}
\newacronym{LMT}{LMT}{Large Message Transfer}
\newacronym{MAGC}{MAGC}{Mapping Approach for GPU Clusters}
\newacronym{MCS}{MCS}{Mellor-Crummey and Scott}
\newacronym{MIG}{MIG}{Multi-Instance GPU}
\newacronym{MIPS}{MIPS}{Million Instructions Per Second}
\newacronym{MPI}{MPI}{Message Passing Interface}
\newacronym{MPP}{MPP}{Massively Parallel Processors}
\newacronym{MPS}{MPS}{Multi-Process Service}
\newacronym{MT-MPI}{MT-MPI}{Multi-Threaded MPI}
\newacronym{Mutex}{Mutex}{Mutual Exclusion}
\newacronym{NBC}{NBC}{Non-Blocking Collective}
\newacronym{NCCL}{NCCL}{NVIDIA Collective Communications Library}
\newacronym{NIC}{NIC}{Network Interface Card}
\newacronym{NPB}{NPB}{NAS Parallel Benchmark suit}
\newacronym{NUMA}{NUMA}{Non-Uniform Memory Access}
\newacronym{NVLink-SLI}{NVLink-SLI}{NVIDIA Scalable Link Interface}
\newacronym{OMB}{OMB}{OSU Micro-Benchmarks}
\newacronym{OMPI}{OMPI}{Open MPI}
\newacronym{OPAL}{OPAL}{Open Portable Access Layer}
\newacronym{OpenACC}{OpenACC}{Open Accelerators}
\newacronym{OpenCL}{OpenCL}{Open Computing Language}
\newacronym{OpenCV}{OpenCV}{Open Computer Vision}
\newacronym{OpenGL}{OpenGL}{Open Graphic Library}
\newacronym{OpenMP}{OpenMP}{Open Multi-Processing}
\newacronym{OpenSHMEM}{OpenSHMEM}{Open Symmetric Hierarchical MEMory}
\newacronym{OS}{OS}{Operating System}
\newacronym{OSC}{OSC}{One-Sided Communication}
\newacronym{P2P}{P2P}{Point-to-Point}
\newacronym{PAP}{PAP}{Process Arrival Pattern}
\newacronym{PAT}{PAT}{Process Arrival Time}
\newacronym{PCIe}{PCIe}{Peripheral Component Interconnect Express}
\newacronym{PGAS}{PGAS}{Partitioned Global Address Space}
\newacronym{PIE}{PIE}{Position-Independent Executables}
\newacronym{POSIX}{POSIX}{Portable Operating System Interface}
\newacronym{PRQ}{PRQ}{Posted Receive Queue}
\newacronym{PRRTE}{PRRTE}{Process Management Interface (PMIx) Reference RunTime Environment}
\newacronym{Pthreads}{Pthreads}{POSIX Threads}
\newacronym{PTX}{PTX}{Parallel Thread Execution}
\newacronym{QE}{QE}{Queue Element}
\newacronym{RAW}{RAW}{Read After Write}
\newacronym{RDMA}{RDMA}{Remote Direct Memory Access}
\newacronym{RMA}{RMA}{Remote Memory Access}
\newacronym{RISC}{RISC}{Reduced Instruction Set Computing}
\newacronym{RSA}{RSA}{Reduce Scatter Allgather}
\newacronym{RTS}{RTS}{Request To Send}
\newacronym{RuntimeAPI}{Runtime API}{Runtime Application Programming Interface}
\newacronym{SHOC}{SHOC}{Scalable Heterogeneous Computing benchmark suite}
\newacronym{SHM}{SHM}{Shared Memory}
\newacronym{SIMD}{SIMD}{Single Instruction Multiple Data}
\newacronym{SIMT}{SIMT}{Single Instruction Multiple Thread}
\newacronym{SM}{SM}{Streaming Multiprocessor}
\newacronym{SMP}{SMP}{Symmetric Multi-Processor}
\newacronym{SQL}{SQL}{Structured Query Language}
\newacronym{TGCS}{TGCS}{Task Graph Computing System}
\newacronym{UCC}{UCC}{Unified Communication Collectives}
\newacronym{UCF}{UCF}{Unified Communication Framework}
\newacronym{UCM}{UCM}{Unified Communication Memory}
\newacronym{UCP}{UCP}{Unified Communication Protocols}
\newacronym{UCS}{UCS}{Unified Communication Services}
\newacronym{UCT}{UCT}{Unified Communication Transports}
\newacronym{UCX}{UCX}{Unified Communication X}
\newacronym{ULT}{ULT}{User-Level Threads}
\newacronym{UM}{UM}{Unified Memory}
\newacronym{UMQ}{UMQ}{Unexpected Message Queue}
\newacronym{UPC}{UPC}{Unified Parallel C}
\newacronym{UVA}{UVA}{Unified Virtual Addressing}
\newacronym{VAS}{VAS}{Virtual Address Space}
\newacronym{VLIW}{VLIW}{Very Long Instruction Word}
\newacronym{VNI}{VNI}{Virtual Network Interface}
\newacronym{WAR}{WAR}{Write After Read}
\newacronym{WAW}{WAW}{Write After Write}
\newacronym{YARN}{Apache YARN}{Yet Another Resource Negotiator}

%% ------------------------------------------------------------------------------%
% *** MATH PACKAGES ***
%
%\usepackage[cmex10]{amsmath}
% A popular package from the American Mathematical Society that provides
% many useful and powerful commands for dealing with mathematics. If using
% it, be sure to load this package with the cmex10 option to ensure that
% only type 1 fonts will utilized at all point sizes. Without this option,
% it is possible that some math symbols, particularly those within
% footnotes, will be rendered in bitmap form which will result in a
% document that can not be IEEE Xplore compliant!
%
% Also, note that the amsmath package sets \interdisplaylinepenalty to 10000
% thus preventing page breaks from occurring within multiline equations. Use:
%\interdisplaylinepenalty=2500
% after loading amsmath to restore such page breaks as IEEEtran.cls normally
% does. amsmath.sty is already installed on most LaTeX systems. The latest
% version and documentation can be obtained at:
% http://www.ctan.org/tex-archive/macros/latex/required/amslatex/math/


% *** SPECIALIZED LIST PACKAGES ***
%
%\usepackage{algorithmic}
% algorithmic.sty was written by Peter Williams and Rogerio Brito.
% This package provides an algorithmic environment fo describing algorithms.
% You can use the algorithmic environment in-text or within a figure
% environment to provide for a floating algorithm. Do NOT use the algorithm
% floating environment provided by algorithm.sty (by the same authors) or
% algorithm2e.sty (by Christophe Fiorio) as IEEE does not use dedicated
% algorithm float types and packages that provide these will not provide
% correct IEEE style captions. The latest version and documentation of
% algorithmic.sty can be obtained at:
% http://www.ctan.org/tex-archive/macros/latex/contrib/algorithms/
% There is also a support site at:
% http://algorithms.berlios.de/index.html
% Also of interest may be the (relatively newer and more customizable)
% algorithmicx.sty package by Szasz Janos:
% http://www.ctan.org/tex-archive/macros/latex/contrib/algorithmicx/


% *** ALIGNMENT PACKAGES ***
%
%\usepackage{array}
% Frank Mittelbach's and David Carlisle's array.sty patches and improves
% the standard LaTeX2e array and tabular environments to provide better
% appearance and additional user controls. As the default LaTeX2e table
% generation code is lacking to the point of almost being broken with
% respect to the quality of the end results, all users are strongly
% advised to use an enhanced (at the very least that provided by array.sty)
% set of table tools. array.sty is already installed on most systems. The
% latest version and documentation can be obtained at:
% http://www.ctan.org/tex-archive/macros/latex/required/tools/


%\usepackage{mdwmath}
%\usepackage{mdwtab}
% Also highly recommended is Mark Wooding's extremely powerful MDW tools,
% especially mdwmath.sty and mdwtab.sty which are used to format equations
% and tables, respectively. The MDWtools set is already installed on most
% LaTeX systems. The lastest version and documentation is available at:
% http://www.ctan.org/tex-archive/macros/latex/contrib/mdwtools/


% IEEEtran contains the IEEEeqnarray family of commands that can be used to
% generate multiline equations as well as matrices, tables, etc., of high
% quality.


%\usepackage{eqparbox}
% Also of notable interest is Scott Pakin's eqparbox package for creating
% (automatically sized) equal width boxes - aka "natural width parboxes".
% Available at:
% http://www.ctan.org/tex-archive/macros/latex/contrib/eqparbox/



% *** SUBFIGURE PACKAGES ***
%\usepackage[tight,footnotesize]{subfigure}
% subfigure.sty was written by Steven Douglas Cochran. This package makes it
% easy to put subfigures in your figures. e.g., "Figure 1a and 1b". For IEEE
% work, it is a good idea to load it with the tight package option to reduce
% the amount of white space around the subfigures. subfigure.sty is already
% installed on most LaTeX systems. The latest version and documentation can
% be obtained at:
% http://www.ctan.org/tex-archive/obsolete/macros/latex/contrib/subfigure/
% subfigure.sty has been superceeded by subfig.sty.



%\usepackage[caption=false]{caption}
%\usepackage[font=footnotesize]{subfig}
% subfig.sty, also written by Steven Douglas Cochran, is the modern
% replacement for subfigure.sty. However, subfig.sty requires and
% automatically loads Axel Sommerfeldt's caption.sty which will override
% IEEEtran.cls handling of captions and this will result in nonIEEE style
% figure/table captions. To prevent this problem, be sure and preload
% caption.sty with its "caption=false" package option. This is will preserve
% IEEEtran.cls handing of captions. Version 1.3 (2005/06/28) and later 
% (recommended due to many improvements over 1.2) of subfig.sty supports
% the caption=false option directly:
%\usepackage[caption=false,font=footnotesize]{subfig}
%
% The latest version and documentation can be obtained at:
% http://www.ctan.org/tex-archive/macros/latex/contrib/subfig/
% The latest version and documentation of caption.sty can be obtained at:
% http://www.ctan.org/tex-archive/macros/latex/contrib/caption/




% *** FLOAT PACKAGES ***
%
%\usepackage{fixltx2e}
% fixltx2e, the successor to the earlier fix2col.sty, was written by
% Frank Mittelbach and David Carlisle. This package corrects a few problems
% in the LaTeX2e kernel, the most notable of which is that in current
% LaTeX2e releases, the ordering of single and double column floats is not
% guaranteed to be preserved. Thus, an unpatched LaTeX2e can allow a
% single column figure to be placed prior to an earlier double column
% figure. The latest version and documentation can be found at:
% http://www.ctan.org/tex-archive/macros/latex/base/


%\usepackage{stfloats}
% stfloats.sty was written by Sigitas Tolusis. This package gives LaTeX2e
% the ability to do double column floats at the bottom of the page as well
% as the top. (e.g., "\begin{figure*}[!b]" is not normally possible in
% LaTeX2e). It also provides a command:
%\fnbelowfloat
% to enable the placement of footnotes below bottom floats (the standard
% LaTeX2e kernel puts them above bottom floats). This is an invasive package
% which rewrites many portions of the LaTeX2e float routines. It may not work
% with other packages that modify the LaTeX2e float routines. The latest
% version and documentation can be obtained at:
% http://www.ctan.org/tex-archive/macros/latex/contrib/sttools/
% Documentation is contained in the stfloats.sty comments as well as in the
% presfull.pdf file. Do not use the stfloats baselinefloat ability as IEEE
% does not allow \baselineskip to stretch. Authors submitting work to the
% IEEE should note that IEEE rarely uses double column equations and
% that authors should try to avoid such use. Do not be tempted to use the
% cuted.sty or midfloat.sty packages (also by Sigitas Tolusis) as IEEE does
% not format its papers in such ways.


% *** PDF, URL AND HYPERLINK PACKAGES ***
%
%\usepackage{url}
% url.sty was written by Donald Arseneau. It provides better support for
% handling and breaking URLs. url.sty is already installed on most LaTeX
% systems. The latest version can be obtained at:
% http://www.ctan.org/tex-archive/macros/latex/contrib/misc/
% Read the url.sty source comments for usage information. Basically,
% \url{my_url_here}.


% *** Do not adjust lengths that control margins, column widths, etc. ***
% *** Do not use packages that alter fonts (such as pslatex).         ***
% There should be no need to do such things with IEEEtran.cls V1.6 and later.
% (Unless specifically asked to do so by the journal or conference you plan
% to submit to, of course. )

